\def\duedate{30.04.2025}
\def\teamname{The Hardcoders}
\def\aufgabenblatt{1}

\documentclass{article}

\usepackage[a4paper, margin=2.5cm]{geometry}
\usepackage{graphicx}
\usepackage[ngerman]{babel} % automatische Silbentrennung
\usepackage[table]{xcolor}
\usepackage{tabularx,array,booktabs,makecell}
\usepackage{titlesec}
\usepackage{amsmath}
\usepackage{tabularx}
\usepackage{xcolor}

\usepackage{fancyhdr}
\pagestyle{fancy} 
\fancyhead[L]{\teamname}  
\fancyhead[C]{Software Engineering\\Aufgabenblatt \aufgabenblatt}
\fancyhead[R]{\\\includegraphics[width=0.25\textwidth]{../common/hs\_aalen\_de.png}}

\fancypagestyle{page1}{
	\fancyhead[L]{\teamname \\Abgabe: \duedate}
	\fancyhead[C]{Software Engineering\\Aufgabenblatt \aufgabenblatt}
	\fancyhead[R]{ \\\includegraphics[width=0.25\textwidth]{../common/hs\_aalen\_de.png}}

}
\setlength{\parindent}{0mm}
\setlength{\parskip}{2.5mm}

\titlespacing*{\section}{0mm}{4pt}{0pt}
\setlength{\headsep}{14mm}

\begin{document}

\thispagestyle{page1} 
In dieser Übung betrachten wir die Anforderungen der Software \textit{Sidequest}. Das Ziel von \textit{Sidequest} ist es Menschen zu helfen ihre Aufgaben zu organisieren.

\section{Aufgabe 1}
Ergänzen Sie pro Gruppenmitglied 1-2 Anwendungsfälle. Hierzu identifizieren Sie Features, in denen Anwendungsfälle sinnvoll sind. Dies sind vor allen die Features, bei denen der Ablauf nicht offensichtlich oder nicht trivial ist. Oft betreffen gute Anwendungfälle mehrere Anforderungen.

\section{Aufgabe 2}
Spezifizieren Sie ein optionales Feature. Die Whatsapp  Integration (O1) dient als Beispiel Anforderungen an das Feature sind: (1) Das Feature soll einen echten Mehrwert liefern mit der realistischerweisen Option, dass Nutzer die Anwendung wegen dieses Features verwenden oder kaufen. (2) Das Feature soll realistisch umsetzbar sein. 

\section{Vision}

\textit{Sidequest} unterstützt Individuen und Gruppen wie z.B. Familien oder Vereine ihre Alltags- und Verwaltungsaufgaben im Griff zu behalten. Durch Künstliche Intelligenz lernt das System, welche Schritte zur Erreichung eines Ziels erforderlich sind und verfeinert Aufgaben (genannt Quests) automatisch zu Schritten (genannt Subquests). \textit{Sidequest} kann erweitert werden, um nach und nach immer mehr Verwaltungsaufgaben automatisch zu übernehmen. 

\section{Ziele}
\begin{itemize}
\item{Ziel 1:} \textit{Sidequest} ist ein Mehrbenutzer System, dass es erlaubt, Aufgaben anzulegen, zu Unteraufgaben zu verfeinern und den Status der Aufgaben zu verfolgen. 
\item{Ziel 2:} \textit{Sidequest} erlaubt es, Aufgaben oder Teilaufgaben einer Aufgabe an andere Personen zu delegieren. 
\item{Ziel 3:} \textit{Sidequest} erkennt, wenn Nutzer wiederkehrend semantisch ähnliche Aufgaben anlegen, und nutzt dies, um automatisch Subquests vorzuschlagen. Diese Erkennung funktioniert sowohl Nutzerbezogen, d.h. das System erkennt, dass ein Nutzer Aufgaben immer auf die gleiche Art erledigt, als auch über Nutzer hinweg, d.h. das System kann eine Aufgabe zerlegen, weil andere Nutzer sie bereits sinnvoll zerlegt haben.
\end{itemize}

\section{Fragen und Probleme}
\begin{itemize}
\item{Frage 1} Wie soll mit Deadlines umgegangen werden?
\item{Frage 2} Wie soll mit Deadlines umgegangen werden?
\end{itemize}

\section{Funktionale Anforderungen}

\begin{tabularx}{0.8\textwidth} { 
  | >{\raggedright\arraybackslash}X 
  | >{\centering\arraybackslash}X 
  | >{\centering\arraybackslash}X 
  | >{\centering\arraybackslash}X 
  | >{\raggedleft\arraybackslash}X | }
 \hline
 Systemqualität & sehr gut & gut & normal & nicht relevant \\
 \hline
 Funktionalität  &   & x &  &   \\
 \hline
 Zuverlässigkeit  &   &  & x &   \\
\hline
 Benutzbarkeit  & x  &  &  &   \\
 \hline
 Effizienz  &   &  & x &   \\
 \hline
 Wartbarkeit  &   & x &  &   \\
 \hline
 Portabilität  & x  &  &  &   \\
 \hline
\end{tabularx}

\subsection{Begriffslexikon}

\begin{samepage}
\textbf{Begriff:} Benutzer \\
\textbf{Bedeutung:} Account eines Benutzers, der von mehreren Endgeräten aus benutzt werden kann. \\
\textbf{Gültigkeit:} Ein \textit{Benutzer} existieren von der Registrierung (siehe R X) eines Benutzers bis zu seiner Löschung (R Y). \\
\textbf{Identifikation:} \textit{Benutzer} wird identifiziert über eine EMail-Adresse. Hierbei findet nur ein Sanity-Check statt (Eingabestring entspricht EMail-Format), aber eine Überprüfung der Existenz eines passenden EMail-Kontos. \\
\end{samepage}

\begin{samepage}
\textbf{Begriff:} Benutzerprofil \\
\textbf{Bedeutung:} Konfigurierbar ist: Zeitraum der automatischen Löschung, ... \\
\textbf{Gültigkeit:} Die Benutzereinstellungen existieren von der Anlage eines Benutzers bis zu seiner Löschung. \\
\textbf{Identifikation:} Ist 1:1 mit dem Benutzer verknüpft. \\
\textbf{Querverweise:} \textit{Benutzer} \\ \\
\end{samepage}

\begin{samepage}
\textbf{Begriff:} Quest \\
\textbf{Bedeutung:} Eine \textit{Quest} repräsentiert eine Aufgabe, die ein \textit{Benutzer} lösen möchte. Es gibt für jede \textit{Quest} einen Besitzer (Standart: Ersteller) und einen Bearbeiter (Standart: Besitzer). \\
\textbf{Abgrenzung:} Quest Template \\
\textbf{Gültigkeit:} Eine \textit{Quest} existiert von ihrer Anlage bis zu ihrer Löschung. \\
\textbf{Identifikation:} Eine \textit{Quest} wird über eine interne ID identifiziert. \\
\textbf{Querverweise:} \textit{Quest Status}, \textit{Haupquest}, \textit{Subquest} \\
\end{samepage}

\begin{samepage}
\textbf{Begriff:} Hauptquest \\
\textbf{Bedeutung:} Eine \textit{Hauptquest} ist eine \textit{Quest} ohne übergeordnete \textit{Quest}. \\
\end{samepage}

\begin{samepage}
\textbf{Begriff:} Besitzer \\
\textbf{Bedeutung:} Der \textit{Besitzer} einer Quest ist ein \textit{Benutzer}, der für eine \textit{Quest} verantwortlich ist und Rechte an dieser in vollem Umfang hat. Es gibt für jede \textit{(Sub-)Quest} nur einen Besitzer. \\
\textbf{Abgrenzung:} Bearbeiter \\
\textbf{Gültigkeit:} Ein \textit{Besitzer} existiert für jede \textit{(Sub-)Quest} von Anfang bis zur Löschung. \\
\textbf{Identifikation:} Der \textit{Besitzer} einer Quest ist ein \textit{Benutzer}. \\
\textbf{Querverweise:} \textit{Bearbeiter} \\ \\
\end{samepage}

\begin{samepage}
\textbf{Begriff:} Bearbeiter \\
\textbf{Bedeutung:} Der \textit{Bearbeiter} einer Quest ist ein \textit{Benutzer}, der berechtigt ist, eine \textit{(Sub-)Quest} zu bearbeiten, d.h. er kann die \textit{Quest} ändern oder ihren Status ändern.\\
\textbf{Abgrenzung:} Besitzer \\
\textbf{Identifikation:} Der \textit{Bearbeiter}  einer Quest ist ein \textit{Benutzer}. \\
\textbf{Querverweise:} \textit{Besitzer} \\ \\
\end{samepage}

\begin{samepage}
\textbf{Begriff:} Gruppe \\
\textbf{Bedeutung:} Eine \textit{Gruppe} ist ein spezieller \textit{Benutzer}, der stellvertretend für eine Menge konkreter \textit{Benutzer} sowohl \textit{Besitzer} als auch \textit{Bearbeiter} sein kann. Anstelle dieses Gruppenbenutzers kann jedes Gruppenmitglied gleichermaßen \textit{Besitzer}- u. \textit{Bearbeiter} verändern und \textit{(Sub-)Quests}, die speziell für diese \textit{Gruppe} erstellt wurden, an Nichtmitglieder delegieren. \\
\textbf{Abgrenzung:} Benutzer \\
\textbf{Gültigkeit:} Eine \textit{Gruppe} existiert von der Erstellung bis zur Löschung. \\
\textbf{Identifikation:} Die \textit{Gruppe} ist ein spezieller \textit{Benutzer} und wird daher auch durch eine interne ID identifiziert. \\
\textbf{Querverweise:} \textit{Benutzer} \\ \\
\end{samepage}

\begin{samepage}
\textbf{Begriff:} Questhierarchie \\
\textbf{Bedeutung:} Eine \textit{Questhierarchie} ist eine Menge an \textit{Quests}, die untereinander verbunden sind. Zur \textit{Questhierarchie} einer \textit{Quest} gehören alle \textit{Subquests}, deren \textit{Subquests} und so weiter sowie alle übergeordneten \textit{Quests} und deren \textit{Questhierarchien}.\\
\textbf{Abgrenzung:} Quest \\
\textbf{Querverweise:} \textit{Quest}, \textit{Haupquest}, \textit{Sidequest} \\ \\
\end{samepage}

\textcolor{blue}
{
\begin{samepage}
\textbf{Begriff:} Rangsystem\\
\textbf{Bedeutung:} Das Rangsystem ermöglicht es \textit{Quests} Punkte zuzuordnen, welche bei Erledigung (R2) vergeben werden\\ 
\textbf{Gültigkeit:} Das Rangsystem für eine Gruppe ist Gültig von der Aktivierung bis zur Deaktivierung.
\textbf{Querverweise:} \textit{Quest}\\
\end{samepage}
}

\begin{samepage}
\textbf{Begriff:} Subquest \\
\textbf{Bedeutung:} Eine \textit{Subquest} ist eine \textit{Quest}, die einer anderen \textit{Quest} untergeordnet ist. Um eine \textit{Quest} zu erfüllen, müssen zunächst alle \textit{Subquests} erfüllt werden. \\
\end{samepage}

\begin{samepage}
\textbf{Begriff:} Quest Status \\
\textbf{Bedeutung:} Eine \textit{Quest} kann verschiedene Status haben. Neu angelegt ist der Status ``offen''. Wird die \textit{Quest} erledigt, ist der Status ``erledigt''. Sind alle Quests einer \textit{Questhierarchie} erledigt, so wird der Status der Hierarchie ``inaktiv', d.h. alle darin enthaltenen Quests werden nicht mehr angezeigt'. Wird eine Quest entfernt, so bekommt sie den Status ``gelöscht''.\\
\textbf{Querverweise:} Quest,  Questhierarchie \\ \\
\end{samepage}

\begin{samepage}
\textbf{Begriff:} Quest Template \\
\textbf{Bedeutung:} Ein \textit{Quest Template} repräsentiert ein wiederkehrendes Ziel, dass \textit{Benutzer} lösen möchten. Durch \textit{Quest Templates} können \textit{Quests} automatisch mit \textit{Subquests} befüllt werden. \\
\textbf{Abgrenzung:} Quest \\
\textbf{Gültigkeit:} Ein \textit{Quest Template} existiert als Programmbestandteil ständig. \\
\textbf{Identifikation:} Eine \textit{Quest Template} wird über eine interne ID identifiziert. \\
\textbf{Querverweise:} Quest, Hauptquest, Sidequest \\ \\
\end{samepage}

\begin{samepage}
\textbf{Begriff:} Weiches Löschen \\
\textbf{Bedeutung:} Es gibt zwei Arten des Löschens. Beim ``weichen Löschen'' (auch Entfernen genannt) wird der Status der Quest auf ``gelöscht'' gesetzt.  \\
\textbf{Abgrenzung:} Hartes Löschen \\
\textbf{Querverweise:} Quest \\ \\
\end{samepage}

\begin{samepage}
\textbf{Begriff:} Hartes Löschen \\
\textbf{Bedeutung:} Es gibt zwei Arten des Löschens. Beim ``harten Löschen'' wird die Quest vollständig aus dem System entfernt.  \\
\textbf{Abgrenzung:} Hartes Löschen \\
\textbf{Querverweise:} Quest \\ \\
\end{samepage}

\subsection{Anforderungen (Requirements)}

\subsubsection{R1 Quest Anlage}

\begin{itemize}
	\item{R1.1} \textit{Benutzer} können \textit{Quests} für sich selber anlegen. Hierzu wird ein Titel und eine optionale Beschreibung eingegeben, \textit{Besitzer} und \textit{Bearbeiter} sind dann der Benutzer selber.
	\item{R1.2} \textit{Benutzer} können \textit{Quests} für eine \textit{Gruppe} anlegen. \textit{Besitzer} und \textit{Bearbeiter} ist dann die \textit{Gruppe}.
	\item{R1.3} \textit{Benutzer} können eine soeben angelegte \textit{Quests} mit \textit{Subquests} befüllen. Das Befüllen einer \textit{Quests} mit potentiell vielen \textit{Subquests} wird in der Benutzerführung so komfortabel wie möglich umgesetzt.
	\item{R1.4} Quests sind grundlegend hierarchisch, d.h. \textit{Subquests} können wieder \textit{Subquests} enthalten. Die Anlage von auch tief gestaffelten \textit{Quests} wird so komfortabel wie möglich umgesetzt.
	\item{R1.5} Bei der Anlage einer \textit{Quests} verwendet das System alle von diesem \textit{Benutzer} bereits verwendeten Titel zur Autocompletion. 
	\item{R1.6} Wenn ein \textit{Benutzer} eine \textit{Quests} erstellt, für die ein \textit{Questtemplate} existiert, so schlägt das System die \textit{Subquests} des \textit{Questtemplate} automatisch vor. Ein \textit{Questtemplate} existiert dann, wenn der Titel einer \textit{Quests} dem Titel
	einer bereits von diesem \textit{Benutzer} erstellten \textit{Quests} exakt entspricht. 
\end{itemize}

\subsubsection{R2 Quest Erledigen}

\begin{itemize}
	\item{R2.1} \textit{Benutzer} können ihnen als \textit{Bearbeiter} zugeordnete \textit{Quests} als ``erledigt'' markieren. 
	\item{R2.2} \textit{Quests} können erst erledigt werden, wenn alle zugehörigen \textit{Subquests} erledigt sind. 
	\item{R2.3} Sind alle \textit{Subquests} einer \textit{Quests} erledigt, wird vorgeschlagen, diese ebenfalls als erledigt zu markieren. 
	\item{R2.4} Sind in einer \textit{Hierarchischen Quests} alle \textit{Quests} erledigt, so heisst die gesamte Hierarchie \textit{inaktiv} und wird ausgeblendet. 
\end{itemize}

\subsubsection{R3 Quest Delegieren}

\begin{itemize}
	\item{R3.1} \textit{Besitzer} von \textit{Quests} können an andere \textit{Benutzer} oder \textit{Gruppen} delegieren. Wird eine \textit{Quests} delegiert, so werden die zugehörigen \textit{Subquests} ebenfalls delegiert. Es gibt zwei Arten von delegieren: wird die Bearbeitung delegiert, so ändert sich der \textit{Bearbeiter} einer \textit{Quest}. Wird die Zuständigkeit delegiert, so ändert sich der \textit{Besitzer}. Delegierung der Bearbeitung ist der Standardfall, dieser muss so komfortabel wie möglich umgesetzt werden. 
	\item{R3.2} Wird eine  \textit{Quest} an einen \textit{Benutzer} delegiert, so muss dieser sie erst annehmen bevor sie ihm zugewiesen wird. 
	\item{R3.3} Ist eine \textit{Gruppen} der \textit{Bearbeiter} einer \textit{Quest}, so kann ein Mitglieder der \textit{Gruppen} die \textit{Quest} oder eine \textit{Subquests} an sich selber delegieren (``ich übernehme das'').
\end{itemize}

\subsubsection{R4 Quest Sichtbarkeit}

\begin{itemize}
	\item{R4.1} \textit{Benutzer} bekommen alle  \textit{Quests} angezeigt, wenn sie entweder \textit{Besitzer} oder \textit{Bearbeiter} sind, entweder direkt oder indirekt über eine \textit{Gruppe}.
	\item{R4.2} Normalerweise werden \textit{Quests} nur angezeigt, wenn sie im Status ``offen'' oder ``erledigt'' sind. \textit{Quests} aus inaktiven  \textit{Hierarchien} oder gelöschte Quests werden nicht angezeigt (Ausnahme: siehe R4.4). 
	\item{R4.3}  \textit{Quests} können in mehreren Ansichten angezeigt werden.

\begin{itemize}
	\item{R4.3.1} Ansicht nach  Priorität
	\item{R4.3.2} Hierarchische Ansicht, gegliedert nach \textit{Hauptquests}
	\item{R4.3.3} Gegliedert nach Besitzer (\textit{Benutzer}  selbst oder Gruppen)
\end{itemize}

	\item{R4.4} \textit{Benutzer} können einen vollständigen Verlauf ihrer  \textit{Quests} anzeigen, dann werden auch Quests im Status ``inaktiv'' oder ``gelöscht'' angezeigt. 
\end{itemize}

\subsubsection{R5 Quest Löschen}

\begin{itemize}
	\item{R5.1} \textit{Benutzer} können \textit{Quests} löschen, wenn sie \textit{Besitzer} sind. Eine \textit{Quest} kann weich gelöscht werden. Das weiche Löschen ist der Standardfall und soll so komfortabel wie möglich umgesetzt werden. Wird eine  \textit{Quest}  gelöscht, so werden alle  \textit{Subquests} ebenfalls gelöscht.
	\item{R5.2} \textit{Quests} können hart gelöscht werden.
	\textcolor{blue}{\item{R5.3} Hatte das zu löschende \textit{Quest} das \textit{Rangsystem} aktiviert, so müssen bei der Löschung entsprechend die vergebenen Punkte angepasst werden.}
\end{itemize}

\subsubsection{R6 Quest Bearbeiten}

\begin{itemize}
	\item{R6.1} \textit{Benutzer} können alle Attribute von \textit{Quests} bearbeiten. Das beinhaltet Titel, Beschreibung, Status.
	\item{R6.2} \textit{Benutzer} können nachträglich \textit{Subquests} ergänzen.
	\item{R6.3} \textit{Benutzer} können \textit{Subquests} hinsichtlich der Reihenfolge innerhalb einer \textit{Quest} nachträglich umsortieren
	\item{R6.4} \textit{Benutzer} können \textit{Subquests} nachträglich einer anderen \textit{Quest} zuordnen.
\end{itemize}

\subsubsection{R7 Benutzerverwaltung}
    \begin{itemize}
        \item{R 7.1} \textit{Benutzer} müssen mit Email Adresse und Namen angelegt werden können.
        \item{R 7.2} \textit{Benutzer} müssen harrt gelöscht werden können.
        \item{R 7.3} Alle Attribute eines \textit{Benutzers} müssen sich ändern lassen können.
        \item{R 7.4} Wird ein \textit{Benutzer} gelöscht, so müssen ihm zugewiesene \textit{Quests} an einen anderen noch existierenen \textit{Benutzer} übertragen werden können. Dieser muss solche \textit{Quests} annehmen können, andernfalls werden sie gelöscht. 
    \end{itemize}

\subsubsection{R8 Gruppenverwaltung}
    \begin{itemize}
        \item{R 8.1} \textit{Benutzer} können \textit{Gruppen} anlegen. Der \textit{Benutzer} ist dann Mitglieder dieser \textit{Gruppe}.
        \item{R 8.2} Bei Anlage einer \textit{Gruppe} können weitere \textit{Benutzer} zugefügt werden. Die neu angelegte \textit{Gruppe} bekommt einen automatisch aus den Namen der Mitglieder gebildeten Namen. 
        \item{R 8.3} \textit{Gruppen} können umbenannt werden. 
        \item{R 8.4} \textit{Gruppen} können wieder Mitglieder in \textit{Gruppen} sein. Es sind keine Zirkel möglich. 
        \item{R 8.5} Ein Mitglied kann das Rangsystem (siehe R9) aktivieren oder deaktivieren
    \end{itemize}
    

       
\subsubsection{O1 Whatsapp  Integration (optional, Tim Dahmen)}
    \begin{itemize}
        \item{O 1.1} Eine Sidequest \textit{Gruppe} kann vom Besitzer der  \textit{Gruppe} mit einer Whatsappgruppe verknüpft werden. 
        \item{O 1.2} Ist eine \textit{Gruppe} mit Whatsapp verknüpft, so sendet das System automatisch Nachrichten in die Whatsappgruppe, wenn in der \textit{Gruppe} eine in der Quest erledigt, geändert, oder angelegt wird.
        \item{O 1.3} Ist eine \textit{Gruppe} mit Whatsapp verknüpft, so sendet das System automatisch Nachrichten in die Whatsappgruppe, wenn ein Benutzer einer Gruppe beitritt oder diese verlässt.
        \item{O 1.4} Die Verknüpfung einer Sidequest \textit{Gruppe} mit einer Whatsappgruppe kann vom Besitzer der  \textit{Gruppe} wieder aufgelöst werden.  
        \item{O 1.5} Ein \textit{Benutzer} kann die Whatsapp Benachrichtigungen pro Quest oder pro Gruppe stumm schalten und die Stummschaltung wieder aufheben. Die Stummschaltung betrifft dann nur diesen Benutzer. 
        \item{O 1.6} Ein \textit{Benutzer} kann sich mit seinen Whatsapp Account verknüpfen und dies wieder aufheben. 
        \item{O 1.7} Ist ein \textit{Benutzer} mit einem Whatsapp Account verknüpft, so erhält der Whatsapp Benutzer Benachrichtigungen wie in O 1.2 und O 1.3 für alle \textit{Quest} und \textit{Gruppen}, mit denen der \textit{Benutzer}  verknüpft ist. 
    \end{itemize}





\subsubsection{O2 Rangsystem in Gruppen (optinal, Marco Bencic, Benjamit Klett, Fabian Strauß)}

      \begin{itemize}
	\item{O 2.1} Der Ersteller einer \textit{Quest} (vgl. R1.1) wird nun außerdem gebeten eine Punktzahl von 1-10 einzugeben, welche beschreiben soll wie schwierig die Quest ist. 			Diese Punktzahl kann nachträglich nur vom Besitzer geändert werden, sie wird in der Questinformation angezeigt. Wird eine Quest durch Subquests verfeinert, wird die Punktzahl 			durch die Summe der Subquestpunkte ersetzt.

	\item{O 2.2} Die Punktzahl wird bei Abschluss einer (Sub-)Quest dem/den Bearbeiter/n gutgeschrieben bzw. gespeichert. Wurde eine (Sub-)Quest an ein Nichtmitglied delegiert, so 		werden niemandem die Punkte gutgeschrieben. \\
  	Bereits erziehlte Punkte werden von der Gesamtpunktanzahl in der Questinformationsanzeige abgezogen, die noch nicht vergebenen Punkte werden von der Gesamtanzahl vom System 			unterschieden.

        \item{O 2.3} In der Profilanzeige eines Benutzers, der in einer Gruppe bzw. einer Gruppenquest als Besitzer oder Bearbeiter erscheint, werden als gruppenspezifische Information 		zusätzlich die Anzahl der zuletzt erreichten Punkte angezeigt. Dies sind standartgemäß die Gesamtpunktzahl der letzten Woche, wahlweise lässt sich aber auch die Anzeige der Punktzahl 		des letzten Monats, Jahres oder der Gesamtpunktanzahl im Profil konfigurieren.
    		
        % Zur Umsetzung: 
        %Eine Gruppe besteht nicht direkt aus Benutzern sondern aus Mitgliedern, die Benutzer mit gruppenspezifischen Zusatzinformationen darstellen. Nur für das bzw. die bearbeitenden 		Mitglieder einer (Sub-)Quest) werden die Punkte bei Abschluss zusätzlich festgehalten (solange sie Gruppenmitglied sind). Festgehalten werden die Quest-ID, der Abschlusszeitpunkt und 		die zugehörigen Punkte

        \item{O 2.4} Die Profilinformation der Gruppe enthält das Gesamtranking der Mitglieder dargestellt als  Tabelle, die den Rang in Form einer Nummerierung, den Namen des Benutzers und 		die zugehörige Punktzahl enthält.
        Wahlweise kann die Anzeige auf wöchentlich, monatlich oder jährlich umgestellt werden, um relative Ergebnisse anzuzeigen. Ein gewünschter Zeitraum kann dann dementsprechend angegeben 		werden.

        \item{O 2.5} Innerhalb einer Gruppe und deren (Sub-)Quests erscheinende Namen von Gruppenmitgliedern erhalten als Zusatzzeichen ein für jeden sichtbares Icon dahinter, wenn sie zu den 	drei Mitgliedern mit der höchsten Gesamtpunktzahl dieser Gruppe gehören (Goldener, silberner und bronzefarbener Pokal). Die Ranginfo besonders fleißiger Mitglieder soll so überall 		innerhalb der Gruppe erkennbar sein.
	
    \end{itemize}

\subsection{Use-Cases}

\begin{samepage}
\textbf{Numbers:} SQ-1\\
\textbf{Name:} Quest abschließen\\
\textbf{Actors:} \textit{Besitzer} der \textit{Quest}, \textit{Bearbeiter} der letzten \textit{Subquest} \\
\textbf{Trigger:} Die letzte offene \textit{Subquest} einer \textit{Quest} wird erledigt.\\
\textbf{Preconditions:} Eine \textit{Quest} (Q\textsubscript{parent}) hat den Status ``offen'' und genau eine \textit{Subquest} (Q\textsubscript{child}), hat ebenfalls den Status ``offen''. \\
\textbf{Postconditions / Goal:} Der Status von Q\textsubscript{parent} ist ``erledigt''. \\
\textbf{Postconditions Special Cases:} 
\begin{itemize}
    \item[3a] Der Status von Q\textsubscript{parent} bleibt ``offen''. 
    \item[4a] Der Status der übergeordneten Quests wird ebenfalls auf ``erledigt'' gesetzt. 
    \item[5a] Der Status der gesamten Questhierarchie wird auf ``inaktiv'' gesetzt.
\end{itemize}
\textbf{Steps Default Case:}
\begin{itemize}
    \item[1] \textit{Bearbeiter} der letzter offener \textit{Subquest} setzt deren Status auf ``erledigt''.
    \item[2] Der \textit{Besitzer} von Q\textsubscript{parent} bekommt eine Nachricht über die Statusänderung und wird gefragt, ob Q\textsubscript{parent} ebenfalls  als ``erledigt'' gesetzt werden soll.
    \item[3] Der \textit{Besitzer} Q\textsubscript{parent} bestätigt die vollständige Bearbeitung der übergeordneten \textit{Quest}.
    \item[4] Es wird überprüft, ob Q\textsubscript{parent} wiederum eine übergeordnete \textit{Quest} besitzt, die die Precondition von SQ-1 erfüllt, so dass sich der Use-Case rekursiv fortsetzt. 
    \item[5] Es wird überprüft, ob in der gesamten \textit{Questhierarchie} von Q\textsubscript{child} keine \textit{Quest} mehr den Status ``offen'' hat (in diesem Fall 5a). 
\end{itemize}
\textbf{Special Cases:}
\begin{itemize}
\item [3a] Besitzer lehnt es ab  Q\textsubscript{parent} auf ``erledigt'' zu setzen.
\item [4a] Weiter mit Schritt 1, wobei Q\textsubscript{parent} die Rolle von Q\textsubscript{child} einnimmt. 
\item [5a] Die gesamte \textit{Questhierarchie} wird auf den Status ``inaktiv'' gesetzt und ausgeblendet. 
\end{itemize}
\end{samepage}

\newpage

\begin{samepage}
	\textbf{Numbers:} SQ-2\\
	\textbf{Name:} Quest löschen\\
	\textbf{Actors:} \textit{Besitzer} der \textit{Quest}, \textit{Benutzer}\\
	\textbf{Trigger:} \textit{Besitzer} löscht (Sub-) Quest\\ 
	\textbf{Preconditions:} Quest existiert \\ 
	\textbf{Postconditions / Goal:} Quest wurde gelöscht / Punkte angepasst\\
	\textbf{Postconditions Special Cases:} 
	\begin{itemize}
		\item Punktzahl wird angepasst 
		\item Löschvorgang verweigert
	\end{itemize}
	\textbf{Steps Default Case:}
	\begin{itemize}
		\item[1] \textit{Besitzer} wählt Quest aus
		\item[2] \textit{Besitzer} wählt Option löschen
		\item[3] \textit{Quest} wird gelöscht 
	\end{itemize}
	\textbf{Special Cases:}
	\begin{itemize}
	\item [2a] Löschvorgang verweigert
	\begin{itemize}
		\item [2a1] Fehlermeldung mit Name des Besitzers der Quest wird ausgegeben 
	\end{itemize} 
	\item [3a] Aktives Rangsystem / Bearbeiter vorhanden
	\begin{itemize}
		\item [3a1] Alle betroffenen Bearbeiter der Quest werden benachrichtigt, dass die Quest gelöscht wird
		\item [3a2] Im Falle eines aktiven Rangsystems werden Punkte von bereits abgeschlossenen Subquests den Bearbeitern, welche diese abgeschlossen haben, wieder abgezogen
	\end{itemize}
	\end{itemize}
	\end{samepage}

\begin{samepage}
	\textbf{Numbers:} SQ-3 \\
	\textbf{Name:} Gruppe erstellen \\
	\textbf{Actors:} \textit{Benutzer}, angebene Gruppenteilnehmer \\
	\textbf{Trigger:} \textit{Benutzer} will eine Neuanlage und wählt den Typ Gruppe aus \\ 
	\textbf{Preconditions:} \textit{Benutzer} existiert und ist eingeloggt \\ 
	\textbf{Postconditions / Goal:} Die neue \textit{Gruppe} existiert im System \\
	\textbf{Postconditions Special Cases:} Die neue \textit{Gruppe} existiert im System \\
	\textbf{Steps Default Case:}
	\begin{itemize}
		\item[1] \textit{Benutzer} wählt Neuanlage, dann Gruppe statt Quest aus.
		\item[2] Eine beliebige Anzahl weiterer \textit{Benutzer} können per Benutzername oder E-Mail eingegeben werden.
		\item[3] Das System überprüft die Benutzerdaten und fügt die \textit{Benutzer} vorerst hinzu.
  		\item[4] Jedem Gruppenmitglied wird eine Nachricht gesendet, die über das Hinzufügen informiert, aber die Ablehnung neben der Kenntnisnahme anbietet.
    		\item[5] Der Benutzer bestätigt, das System generiert einen Gruppennamen aus den Namen der Mitglieder und legt die Gruppe.
      		\item[6] Die anderen Mitglieder lehnen das Hinzufügen nicht ab und sind weiterhin Teil der Gruppe.
	\end{itemize}
	\textbf{Special Cases:}
	\begin{itemize}
	\item [3a] Die Überprüfung schlägt fehl, die nicht gefundenen Benutzernamen/E-Mails werden entfernt und ein Fehlerhinweis wird ausgegeben. Weiter mit Schritt 2.
 	\item [5b] Der Benutzer gibt einen eigenen Gruppennamen ein und bestätigt, in dem Fall bleibt der Name bestehen und das System legt die Gruppe an.
  	\item [6c] Ein oder mehrere Mitglieder lehnen die Teilnahme ab, sie werden dann aus der Gruppe gelöscht.
	\begin{itemize}
		\item [2a1] Fehlermeldung mit Name des Besitzers der Quest wird ausgegeben 
	\end{itemize} 
	\item [3a] Aktives Rangsystem / Bearbeiter vorhanden
	\begin{itemize}
		\item [3a1] Alle betroffenen Bearbeiter der Quest werden benachrichtigt, dass die Quest gelöscht wird
		\item [3a2] Im Falle eines aktiven Rangsystems werden Punkte von bereits abgeschlossenen Subquests den Bearbeitern, welche diese abgeschlossen haben, wieder abgezogen
	\end{itemize}
	\end{itemize}
	\end{samepage}

\newpage
\section{Nicht Funktionale Anforderungen}

\subsubsection{NF1 Systemumgebung}

\begin{itemize}
	\item{NF1.1} Das System soll eine unter Windows lauffähige App aufweisen. 
	\item{NF1.2} Das System soll eine Android App aufweisen (optional, Tim Dahmen). 
	\item{NF1.3} Das System soll eine Apple App aufweisen (optional, Tim Dahmen). 
\end{itemize}

\subsubsection{NF2 Skalierbarkeit}

\begin{itemize}
	\item{NF2.1} Ein \textit{Benutzer} kann hunderte von \textit{Quests} verwalten, die nicht im Status ``inaktiv'' sind.
	\item{NF2.2} Ein \textit{Benutzer} kann tausende von \textit{Quests} verwalten, die im Status ``inaktiv'' sind.
	\item{NF2.3} Das System kann Millionen \textit{Quests} verwalten.
	\item{NF2.4} Eine \textit{Gruppe} kann hunderte \textit{Benutzer} enthalten.
	\item{NF2.5} Ein \textit{Quests} kann maximal hundert direkte \textit{Subquests} enthalten.
	\item{NF2.6} Ein \textit{Questhierarchie} kann tausende \textit{Quests} enthalten.
	\item{NF2.7} Das System kann tausende \textit{Benutzer} enthalten.
\end{itemize}

\subsubsection{NF3 Performanz}

\begin{itemize}
	\item{NF3.1} Das System ist jederzeit ohne als störend wahrgenommene Verzögerung bedientbar. Es gibt die Vermutung, dass dies bei einer Antwortzeit <1500ms für Transaktionen der Fall ist. 
	\item{NF3.2} Das System kann bis zu 10 Änderungen pro Sekunde verarbeiten. 
	\item{NF3.3} Das System kann mit maximal tausend gleichzeitg eingeloggten  \textit{Benutzer} umgehen. 
\end{itemize}

\end{document}
