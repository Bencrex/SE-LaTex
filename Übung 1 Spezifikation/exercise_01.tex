\def\duedate{09.04.2025}
\def\teamname{The Hardcoders}
\def\aufgabenblatt{1}

\documentclass{article}

\usepackage[a4paper, margin=2.5cm]{geometry}
\usepackage{graphicx}
\usepackage[ngerman]{babel} % automatische Silbentrennung
\usepackage[table]{xcolor}
\usepackage{tabularx,array,booktabs,makecell}
\usepackage{titlesec}
\usepackage{amsmath}
\usepackage{tabularx}

\usepackage{fancyhdr}
\pagestyle{fancy} 
\fancyhead[L]{\teamname}  
\fancyhead[C]{Software Engineering\\Aufgabenblatt \aufgabenblatt}
\fancyhead[R]{\\\includegraphics[width=0.25\textwidth]{../common/hs_aalen_de.png}}

\fancypagestyle{page1}{
	\fancyhead[L]{\teamname \\Abgabe: \duedate}
	\fancyhead[C]{Software Engineering\\Aufgabenblatt \aufgabenblatt}
	\fancyhead[R]{ \\\includegraphics[width=0.25\textwidth]{../common/hs_aalen_de.png}}

}
\setlength{\parindent}{0mm}
\setlength{\parskip}{2.5mm}

\titlespacing*{\section}{0mm}{4pt}{0pt}
\setlength{\headsep}{14mm}

\begin{document}

\thispagestyle{page1} 
In dieser Übung betrachten wir die Anforderungen der Software \textit{Sidequest}. Das Ziel von \textit{Sidequest} ist es Menschen zu helfen ihre Aufgaben zu organisieren.


\section{Team Setup}

Please group yourself in teams of 3-4 persons. Agree on a team name. 

\textbf{Team Name:} \teamname

\textbf{Team Mitglieder:}
\begin{enumerate}
\item Marco Bencic (3005314)
\item Benjamin Klett (3003345)
\item Fabian Strauß (3003662)
\end{enumerate}

\section{Probleme}

\section{Fragen}

\begin{itemize}
	\item Wie soll das System mit bestehender oder fehlender Onlineverbindung umgehen (Benji Klett)?
    \item Soll jeder \textit{Plugins} zur Verfügung stellen können oder sollen die verfügbaren \textit{Plugins} reguliert werden? (Marco Bencic)
    \item Soll die Anzahl der \textit{Benutzer}, die die \textit{Subquests} eines \textit{Quests} erledigen auf ein Maximum beschränkt werden ?
\end{itemize}

\section{Vision}

\textit{Sidequest} unterstützt Individuen und Gruppen wie z.B. Familien ihre Alltags- und Verwaltungsaufgaben im Griff zu behalten. Durch Künstliche Intelligenz lernt das System , welche Schritte zur Erreichung eines Ziels erforderlich sind und verfeinert Aufgaben (genannt Quests) automatisch zu Schritten (genannt Subquests). \textit{Sidequest} kann erweitert werden, um nach und nach immer mehr Verwaltungsaufgaben automatisch zu übernehmen. 

\section{Ziele}

\textit{Sidequest} ist ein Mehrbenutzer System, dass es erlaubt, Aufgaben anzulegen, zu Unteraufgaben zu verfeinern und den Status der Aufgaben zu verfolgen. Es unterstützt, Teilaufgaben einer Aufgabe an andere Personen zu delegieren. 

\textit{Sidequest} unterstützt die Benutzer unterwegs und auf der Arbeit.

\textit{Sidequest} erkennt, wenn Nutzer wiederkehrend semantisch ähnliche Aufgaben anlegen, und nutzt dies, um automatisch Subquests vorzuschlagen. Diese Erkennung funktioniert sowohl Nutzerbezogen, d.h. das System erkennt, dass ein Nutzer Aufgaben immer auf die gleiche Art erledigt, als auch über Nutzer hinweg, d.h. das System kann eine Aufgabe zerlegen, weil andere Nutzer sie bereits sinnvoll zerlegt haben.

\textit{Sidequest} verfügt über ein Plugin-System, welches Entwicklern die Möglichkeit gibt, den Funktionsumfang zu erweitern. (Marco Bencic)

\section{Functional Requirements}

\begin{tabularx}{0.8\textwidth} { 
  | >{\raggedright\arraybackslash}X 
  | >{\centering\arraybackslash}X 
  | >{\centering\arraybackslash}X 
  | >{\centering\arraybackslash}X 
  | >{\raggedleft\arraybackslash}X | }
 \hline
 Systemqualität & sehr gut & gut & normal & nicht relevant \\
 \hline
 Funktionalität  &   & x &  &   \\
 \hline
 Zuverlässigkeit  &   &  & x &   \\
\hline
 Benutzbarkeit  & x  &  &  &   \\
 \hline
 Effizienz  &   &  & x &   \\
 \hline
 Wartbarkeit  &   & x &  &   \\
 \hline
 Portabilität  & x  &  &  &   \\
 \hline
\end{tabularx}

\subsection{Dictionary}

\begin{samepage}
\textbf{Begriff:} Benutzer \\
\textbf{Bedeutung:} Account eines Systembenutzers, der von mehreren Endgeräten aus benutzt werden kann. \\
\textbf{Gültigkeit:} Ein \textit{Benutzer} existieren von der Registrierung (R X) eines Benutzers bis zu seiner Löschung (R Y). \\
\textbf{Identifikation:} EMail-Adresse, nur Sanity-Check (Eingabestring entspricht EMail-Format). Keine Überprüfung der Existenz eines passenden EMail-Kontos. \\
\end{samepage}

\begin{samepage}
\textbf{Begriff:} Benutzereinstellungen \\
\textbf{Bedeutung:} Konfigurierbar ist: Zeitraum der automatischen Löschung, ... \\
\textbf{Gültigkeit:} Die Benutzereinstellungen existieren von der Anlage eines Benutzers bis zu seiner Löschung. \\
\textbf{Identifikation:} Ist 1:1 mit dem Benutzer verknüpft. \\
\textbf{Querverweise:} \textit{Benutzer} \\ \\
\end{samepage}

\begin{samepage}
\textbf{Begriff:} Quest \\
\textbf{Bedeutung:} Eine \textit{Quest} repräsentiert eine Aufgabe, die ein \textit{Benutzer} lösen möchte. Es gibt für jede \textit{Quest} einen Besitzer (Standart: Ersteller) und einen Bearbeiter (Standart: Besitzer). \\
\textbf{Abgrenzung:} Quest Template \\
\textbf{Gültigkeit:} Eine \textit{Quest} existiert von ihrer Anlage bis zu ihrer Löschung. \\
\textbf{Identifikation:} Eine \textit{Quest} wird über eine interne ID identifiziert. \\
\textbf{Querverweise:} \textit{Haupquest}, \textit{Sidequest} \\ \\
\end{samepage}

\begin{samepage}
\textbf{Begriff:} Hauptquest \\
\textbf{Bedeutung:} Eine \textit{Hauptquest} ist eine \textit{Quest} ohne übergeordnete \textit{Quest}. \\
\end{samepage}

\begin{samepage}
\textbf{Begriff:} Besitzer \\
\textbf{Bedeutung:} Der \textit{Besitzer} einer Quest ist ein \textit{Benutzer}, der für eine \textit{Quest} verantwortlich ist und Rechte an dieser in vollem Umfang hat. Es gibt für jede \textit{(Sub-)Quest} nur einen Besitzer. Dieser kann den \textit{Bearbeiter} abändern bzw. neu zuteilen durch Anfragen. \\
\textbf{Abgrenzung:} Bearbeiter \\
\textbf{Gültigkeit:} Ein \textit{Besitzer} existiert für jede \textit{(Sub-)Quest} von Anfang bis zur Löschung dieser. \\
\textbf{Identifikation:} Der \textit{Besitzer} ist einer Quest fest zugeordnet. \\
\textbf{Querverweise:} \textit{Bearbeiter} \\ \\
\end{samepage}

\begin{samepage}
\textbf{Begriff:} Bearbeiter \\
\textbf{Bedeutung:} Der \textit{Bearbeiter} einer Quest ist ein \textit{Benutzer}, der berechtigt ist, eine \textit{(Sub-)Quest} zu bearbeiten. Es kann mehrere Bearbeiter für eine \textit{(Sub-)Quest} geben. Diese haben keine Übertragungsrechte, eine Anfrage zum Erlangen des Bearbeiterstatus kann nur vom \textit{Besitzer} angenommen oder abgelehnt werden.
 \\
\textbf{Abgrenzung:} Besitzer \\
\textbf{Gültigkeit:} Ein \textit{Bearbeiter} existiert für jede \textit{(Sub-)Quest} von Anfang bis zur Löschung dieser . \\
\textbf{Identifikation:} Der \textit{Bearbeiter} ist einer Quest fest zugeordnet. \\
\textbf{Querverweise:} \textit{Besitzer} \\ \\
\end{samepage}

\begin{samepage}
\textbf{Begriff:} Gruppe \\
\textbf{Bedeutung:} Eine \textit{Gruppe} ist ein spezieller \textit{Benutzer}, der stellvertretend für eine Menge konkreter \textit{Benutzer} sowohl \textit{Besitzer} als auch \textit{Bearbeiter} sein kann. Anstelle dieses Gruppenbenutzers kann jedes Gruppenmitglied gleichermaßen \textit{Besitzer}- u. \textit{Bearbeiter} verändern und \textit{(Sub-)Quests}, die speziell für diese \textit{Gruppe} erstellt wurden, an Nichtmitglieder delegieren.
 \\
\textbf{Abgrenzung:} Benutzer \\
\textbf{Gültigkeit:} Eine \textit{Gruppe} existiert von der Erstellung bis zur Löschung dieser. \\
\textbf{Identifikation:} Die \textit{Gruppe} ist ein spezieller \textit{Benutzer} und wird daher auch durch eine interne ID identifiziert. \\
\textbf{Querverweise:} \textit{Benutzer} \\ \\
\end{samepage}

\begin{samepage}
\textbf{Begriff:} Hierarchisches Quest \\
\textbf{Bedeutung:} Eine \textit{hierarchisches Quest} ist ein \textit{Quest}, das \textit{Subquests} besitzt, die in einer bestimmten Reihenfolge abgearbeitet werden müssen.  \\
\textbf{Abgrenzung:} Quest \\
\textbf{Identifikation:} Eine \textit{hierarchisches Quest} wird über eine interne ID identifiziert. \\
\textbf{Querverweise:} \textit{Haupquest}, \textit{Sidequest} \\ \\
\end{samepage}

\begin{samepage}
\textbf{Begriff:} Plugin \\
\textbf{Bedeutung:} Ein \textit{Plugin} ist eine nachträglich installierte Software, welche den Funktionsumfang von \textit{Sidequest} erweitert. \\
\textbf{Abgrenzung:} Standard Optionen \\
\textbf{Gültigkeit:} Damit ein \textit{Plugin} funktioniert, muss dieses installiert und aktiviert werden  \\
\textbf{Identifikation:} Eine \textit{Plugin} wird über seinen Herausgeber identifiziert \\
\end{samepage}

\begin{samepage}
\textbf{Begriff:} Subquest \\
\textbf{Bedeutung:} Eine \textit{Subquest} ist eine \textit{Quest}, die einer anderen \textit{Quest} untergeordnet ist. Um eine \textit{Quest} zu erfüllen, müssen zunächst alle \textit{Subquests} erfüllt werden. \\
\end{samepage}

\begin{samepage}
\textbf{Begriff:} Quest Template \\
\textbf{Bedeutung:} Ein \textit{Quest Template} repräsentiert ein wiederkehrendes Ziel, dass \textit{Benutzer} lösen möchten. Durch \textit{Quest Templates} können \textit{Quests} automatisch mit \textit{Subquests} befüllt werden. \\
\textbf{Abgrenzung:} Quest \\
\textbf{Gültigkeit:} Ein \textit{Quest Template} existiert als Programmbestandteil ständig. \\
\textbf{Identifikation:} Eine \textit{Quest Template} wird über eine interne ID identifiziert. \\
\textbf{Querverweise:} Quest, Hauptquest, Sidequest \\ \\
\end{samepage}

\subsection{Requirements}

\begin{itemize}
\item R1 Ändern von Zuständigkeiten für \textit{Quests} (Fabian Strauß)
    \begin{itemize}
        \item R1.1 Jede \textit{Quest} bzw. \textit{Subquest} hat sowohl Besitzer als auch Bearbeiter, beide können unabhängig von einander an andere Nutzer übertragen werden.
	\item R1.2 Die Übertragung eine \textit{(Sub-)Quest} kann sowohl beim aktuellen 	Besitzer als auch beim Bearbeiter angefragt werden, diese können die Anfrage annehmen oder ablehnen (User werden bekannt durch peröhnliches Austauschen der Profildaten oder durch Mitgliedschaft in einer gemeinsamen Gruppe).
 	\item R1.3 Der aktuelle Besitzer/Bearbeiter kann auch einen anderen User um Übernahme einer \textit{(Sub-)Quest} bitten, welche diese annehmen können.
    \end{itemize}    

\item R2 Löschen von \textit{Quests} (Datenschutzbeauftragter) 
	\begin{itemize}
	\item R 2.1 \textit{Benutzer} können \textit{Quests} manuell löschen, und zwar einzeln (unabhängig vom Status) oder in ganzen Zeiträumen (nur abgeschlossene Quests). (Tom Gallo, Josa Kirscher) 
	\item R 2.2 Abgeschlossene \textit{Quests} werden automatisch nach einem bestimmten Zeitraum gelöscht, wenn dies in den \textit{Benutzereinstellungen} so konfiguriert ist. (Tom Gallo, Josa Kirscher)
	\end{itemize}

\item \textit{Benutzer} können einen vollständigen Verlauf ihrer Quests anschauen (Tim Dahmen) 

\item \textit{Benutzer} können Quests manuell anlegen. (Tim Dahmen) 
\item Die manuelle Anlage von Quests und Subquests soll so einfach und schnell wie möglich erfolgen. (Tim Dahmen) 
\item Quests können verschiedene Status haben. (Tim Dahmen) 

\item R3 Benutzerverwaltung (Datenschutzbeauftragter) (Marco Bencic)
    \begin{itemize}
        \item R 3.1 \textit{Benutzer} müssen mit Email Adresse und Namen angelegt werden können.
        \item R 3.2 \textit{Benutzer} müssen gelöscht werden können
        \item R 3.3 Gelöschten \textit{Benutzern} zugewiesene \textit{Quests} müssen an einen anderen noch existierenen \textit{Benutzer} übertragen werden können. Dieser muss solche \textit{Quests} annehmen können, andernfalls werden sie gelöscht. 
    \end{itemize}
\item R4 Installation/Deinstallation von Plugins (IT-Abteilung) (Marco Bencic)
    \begin{itemize}
        \item R 4.1 \textit{Plugins} müssen direkt in \textit{Sidequest} installiert und deinstalliert werden können 
        \item R 4.2 Es muss eine Option geben, in der man aktiv zustimmen muss, dass man \textit{Plugins} verwenden will
        \item R 4.3 Bei Neuinstallation und beim Erscheinen neuer Plugins soll eine Popup-Nachricht beim Öffnen der App den \textit{Benutzer} darauf hinweisen, dass diese existieren und zur Installation bereit stehen. (Fabian Strauß)
    \end{itemize}

\item R5 Subquests
    \begin{itemize}
        \item R 5.1 \textit{Benutzer} können Quests manuell in Subquests zerlegen. (Tim Dahmen) 
        \item R 5.2 \textit{Subquests} können wiederum in weitere \textit{Subquests} zerlegt werden (nested \textit{Subquests}) (Marco Bencic)
    \end{itemize}
\item R6 Questvergabe (Marco Bencic)
    \begin{itemize}
        \item R 6.1 \textit{Quests} müssen verschiedenen \textit{Benutzern} zugewiesen werden können
        \item R 6.2 \textit{Subquests} können auf mehrere Benutzer aufgeteilt werden sofern diese nicht Teil eines \textit{hierarchischen Quests} sind
    \end{itemize}
\item R7 Benutzereinstellungen (Datenschutzbeauftragte) (Benji Klett)
    \begin{itemize}
        \item R 7.1 Der Name und die E-Mailadresse eines \textit{Benutzer}s müssen sich ändern lassen
        \item R 7.2 Ein \textit{Benutzer} kann die Einwilligung der Datenverarbeitung durch KI zurückziehen
    \end{itemize}

\begin{itemize}
\item R8 Anlage von \textit{Gruppen} zur gemeinsamen Bearbeitung von \textit{Quests} (Fabian Strauß)
    \begin{itemize}
        \item R1.1 Jeder \textit{Benutzer} soll \textit{Gruppen} anlegen können. \textit{(Sub-)Quests} können dann speziell für diese \textit{Gruppe} zur Bearbeitung erstellt werden.
	\item R1.2 Im Gegensatz zu normalen \textit{(Sub-)Quests} sollen alle Gruppenmitglieder Besitzerrechte ausüben und delegieren können, \textit{Besitzer} bleibt jedoch bei Gruppenquests immer die \textit{Gruppe}. Die Bearbeitung kann auch an Nichtmitglieder delegiert werden.
    \end{itemize}

    
\item R9 Belohnungssystem (Fabian Strauß)
    \begin{itemize}
        \item R 9.1 \textit{Quests} bzw. \textit{Subquests} müssen Belohnungspunkte zugewiesen werden können, auf dessen Basis die App solche Punkte für andere \textit{Benutzer} bei Erledigung vergibt. Die KI-Unterstützung soll eine Klassifizierung der (Unter-)Aufgaben vornehmen, um die automatische Punktevergabe zu optimieren.
        \item R 9.2 \textit{Benutzer} sollen zusätzlich progressiv steigende Boni auf erledigte Aufgaben bekommen können, wenn diese zb. täglich erledigt werden mit Beginn von vorne, falls ein Benutzer die Kette unterbricht.
        \item R 9.3 Ein für jeden \textit{Benutzer} sichtbares Ranking soll existieren, dass die \textit{Benutzer} mit den höchsten Punktzahlen dort auflistet.
    \end{itemize}

\end{itemize}
\end{itemize}
\subsection{Use-Cases}

\begin{samepage}
\textbf{Numbers:} SQ-3.1\\
\textbf{Name:} Quest abschließen\\
\textbf{Actors:} Besitzer der Quest, Bearbeiter der letzten Subquest\\
\textbf{Trigger:} Die letzte unbeendete Subquest wird beendet\\
\textbf{Preconditions:} Der Status von mindestens einer Subquest ist nicht abgeschlossen\\
\textbf{Postconditions / Goal:} Der Status aller Subquests und damit der Status der Quest ist abgeschlossen, die Quest steht nicht mehr zur Bearbeitung bereit (Der Status kann nicht mehr geändert werden). \\
\textbf{Postconditions Special Cases:} 
\begin{itemize}
    \item Besitzer akzeptiert die erfolgreiche Bearbeitung eines Bearbeiters einer Subquest nicht, Status der Quest und der Subquest stehen dann immer noch auf offen.\\
\end{itemize}
\textbf{Steps Default Case:}
\begin{enumerate}
    \item Bearbeiter letzter offener Subquest setzt deren Status auf abgeschlossen
    \item Besitzer bekommt eine Nachricht über die Statusänderung
    \item Besitzer bestätigt die vollständige Bearbeitung der Subquest
    \item Die Subquest und die gesamte Quest hat jetzt den Status abgeschlossen
    \item Beteiligte Benutzer bekommen eine Nachricht, die auf die erfolgreiche Bearbeitung der Quest hinweist.
\end{enumerate}
\textbf{Special Cases:}
\begin{itemize}
\item [3a] Besitzer lehnt die erfolgreiche Bearbeitung der letzten offenen
\begin{itemize}
    \item [3a1] Die Anfrage wird auf abgelehnt gesetzt.
    \item [3a2] Der antragstellende Bearbeiter bekommt eine Mitteilung über die Ablehnung seiner Anfrage.
    Subquest ab.
\end{itemize}
\end{itemize}
\end{samepage}

\newpage
\begin{samepage}
\textbf{Numbers:} SQ-1.1\\
\textbf{Name:} Quest erstellen\\
\textbf{Actors:} Benutzer\\
\textbf{Trigger:} Benutzer möchte ein neues Quest hinzufügen\\
\textbf{Preconditions:}  App befindet sich im Hauptmenü\\
\textbf{Postconditions / Goal:} Quest wird erstellt und angezeigt\\
\textbf{Postconditions Special Cases:}
\begin{itemize}
    \item Erstellung verweigert, wegen unzulässigem Namen des Quests
    \item Erstellung verweigert, wegen fehlenden Informationen
\end{itemize}
\textbf{Steps Default Case:} 
\begin{enumerate}
    \item Benutzer klickt auf Quest hinzufügen
    \item Benutzer gibt Daten bezüglich des erstellten Quests ein 
    \item Eingegebene Daten werden überprüft 
    \item Benutzer speichert das Quest
\end{enumerate}
\textbf{Special Cases:}
\begin{itemize}
    \item [3a] Ungültiger Questname (unerlaubte Zeichen oder Name bereits vorhanden)
    \begin{itemize}
        \item [3a1] Anzeigen von Fehlermeldung, dass Questname nicht zulässig ist
        \item [3a2] Questerstellungsbildschirm wird angezeigt
    \end{itemize}
    \item [4a] Fehlende Informationen
    \begin{itemize}
        \item [4a1] Markieren von Textboxen von fehlenden Informationen
    \end{itemize}
\end{itemize}
\end{samepage}

\newpage
\begin{samepage}
\textbf{Numbers:} SQ-3.2\\
\textbf{Name:} Benutzer registrieren\\
\textbf{Actors:} Anwender\\
\textbf{Trigger:} Anwender installiert die App, öffnet diese und fordert eine Registrierung an.\\
\textbf{Preconditions:} Das gewünschte Benutzerkonto existiert nicht im System. \\
\textbf{Postconditions / Goal:} Ein Benutzerkonto existiert für den Anwender im System.\\
\textbf{Postconditions Special Cases:} 
\begin{itemize}
    \item Es wurde ein neues Konto erstellt
\end{itemize}
\textbf{Steps Default Case:}
\begin{enumerate}
    \item Anwender fordert eine Registrierung im System an
    \item System zeigt Eingabemaske für die erforderlichen Profilangaben
    \item Anwender gibt Benutzernamen, E-mail-Adresse und optional einen Vor- bzw. einen Nachnamen ein und bestätigt die Eingabe.
    \item Ein Hinweis über die erfolgreiche Erstellung des Kontos wird angezeigt und das Hauptmenü wird angezeigt, welches sich im in das Konto eingeloggten Zustand befindet.
\end{enumerate}
\textbf{Special Cases:}
\begin{itemize}
\item [3a] Benutzername oder E-Mail existieren bereits, E-Mail besteht den Sanity-Check nicht.
\begin{itemize}
    \item [3a1] Anwender bekommt entsprechende Fehlermeldung.
    \item [3a2] Bestätigung bleibt aus, Anwender verbleibt in der Registrierungsansicht.
    Subquest ab.
\end{itemize}
\end{itemize}
\end{samepage}

\newpage
\begin{samepage}
    \textbf{Numbers:} SQ-1.2\\
    \textbf{Name:} Quest löschen\\
    \textbf{Actors:} Benutzer\\
    \textbf{Trigger:} Benutzer möchte ein bestehendes Quest löschen\\
    \textbf{Preconditions:}  App befindet sich im Hauptmenü\\
    \textbf{Postconditions / Goal:} Quest wird gelöscht \\
    \textbf{Postconditions Special Cases:} 
    \begin{itemize}
        \item Quest kann nicht gelöscht werden, da Benutzer nicht owner des Quests ist 
    \end{itemize}
    \textbf{Steps Default Case:} 
    \begin{enumerate}
        \item Benutzer wählt Quest aus, das gelöscht werden soll
        \item Benutzer klickt auf Quest löschen
        \item Hauptmenü wird angezeigt
    \end{enumerate}
    \textbf{Special Cases:}
    \begin{itemize}
        \item [2a] Fehlende Berechtigungen 
        \begin{itemize}
            \item [2a1] Fehlermeldung, da Benutzer nicht owner des Quests ist
            \item [2a2] Hauptmenü wird angezeigt
        \end{itemize}
    \end{itemize}
\end{samepage}

\newpage
\begin{samepage}
    \textbf{Numbers:} SQ-1.3\\
    \textbf{Name:} Benutzer löschen\\
    \textbf{Actors:} Benutzer\\
    \textbf{Trigger:} Benutzer möchte einen anderen Benutzer löschen\\
    \textbf{Preconditions:}  App befindet sich in der Benutzerverwaltung\\
    \textbf{Postconditions / Goal:} Löschung eines Benutzers\\
    \textbf{Postconditions Special Cases:}
    \begin{itemize}
        \item Benutzer kann nicht gelöscht werden, da er noch aktiven Quests zugeteilt ist
        \item Benutzer kann nicht gelöscht werden, da er nicht die notwendigen Berechtigungen hat
    \end{itemize}
    \textbf{Steps Default Case:} 
    \begin{enumerate}
        \item Benutzer wählt den zu löschenden Benutzer aus
        \item Benutzer betätigt löschen
        \item Benutzer bestätigt den Löschvorgang
    \end{enumerate}
\textbf{Special Cases:}
\begin{itemize}
    \item [2a] Löschvorgange Fehlgeschlagen (aktive Quests)
    \begin{itemize}
        \item [2a1] Fehlermeldung, dass Benutzer aufgrund von aktiven Quests nicht gelöscht werden kann
        \item [2a2] Anzeigen von Quests, welche für den Fehlschlag des Löschvorganges verantwortlich sind
    \end{itemize}
    \item [2b] Fehlende Berechtigungen
    \begin{itemize}
        \item [2b1] Fehlermeldung, dass Benutzer aufgrund von fehlenden Berechtigungen nicht gelöscht werden kann
        \item [2b2] Bildschirm Benutzerverwaltung wird angezeigt 
    \end{itemize}
\end{itemize}
\end{samepage}

\newpage
\begin{samepage}
    \textbf{Numbers:} SQ-1.4\\
    \textbf{Name:} Plugin installieren\\
    \textbf{Actors:} Benutzer\\
    \textbf{Trigger:} Benutzer möchte ein Plugin istallieren\\
    \textbf{Preconditions:}  Benutzer befindet sich im Plugin Browser\\
    \textbf{Postconditions / Goal:} Erfolgreiche Installation von Plugin\\
    \textbf{Postconditions Special Cases:}
    \begin{itemize}
        \item Installation nicht möglich aufgrund von fehlender Internetverbindung
        \item Installation von Unbekannten Quellen wurde nicht zugestimmt
    \end{itemize}
    \textbf{Steps Default Case:} 
    \begin{enumerate}
        \item Benutzer öffnet Plugin Browser
        \item Benutzer benutzt Suchleiste um gewünschtes Plugin zu suchen
        \item Suchergebnise werden dargestellt 
        \item Benutzer wählt gewünschtes Plugin aus
        \item Benutzer wählt installieren
    \end{enumerate}
\textbf{Special Cases:}
\begin{itemize}
    \item [1a] Fehlende Zustimmung
    \begin{itemize}
        \item [1a1] Fehlermeldung, dass Installationen von unbekannten Quellen nicht zugestimmt wurde
    \end{itemize}
    \end{itemize}
        
\begin{itemize}
    \item [2a] Fehlende Internetverbindung  
    \begin{itemize}
        \item [2a1] Fehlermeldung, dass keine Internetverbindung vorhanden ist 
        \item [2a2] Anzeigen von Plugin Browser
    \end{itemize}
\end{itemize}
\end{samepage}

\newpage
\section{Non-Functional Requirements}

\end{document}
